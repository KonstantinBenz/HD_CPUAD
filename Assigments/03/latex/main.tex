% !TEX root = ./main.tex
% !TEX encoding = UTF-8 Unicode
\documentclass[a4paper,12pt]{article}

% ==================================================
% Sprache und Zeichencodierung
\usepackage[german]{babel}
\usepackage[utf8]{inputenc}
\usepackage[T1]{fontenc}

% ==================================================
% Seitenlayout
\usepackage[
  left=2.5cm,
  right=2.5cm,
  top=2.5cm,
  bottom=2.5cm
]{geometry}

% ==================================================
% Mathematik
\usepackage{amsmath}
\usepackage{amsfonts}
\usepackage{amssymb}
\usepackage{amsthm}
\usepackage{physics}
\usepackage{mathrsfs}
\usepackage{dsfont}
\usepackage{esint}

% ==================================================
% Programmierung
\usepackage{listings}

% ==================================================
% Textgestaltung
\usepackage{enumerate}
\usepackage[shortlabels]{enumitem}
\usepackage{framed}
\usepackage{csquotes} % wichtig für deutsche Anführungszeichen
\usepackage{microtype} % bessere Silbentrennung und Schriftbild

% ==================================================
% Tabellen, Grafiken
\usepackage{float}
\usepackage{tabularx}
\usepackage{multicol}
\usepackage{caption}
\usepackage{subcaption}
\captionsetup{
    format=hang,
    margin=10pt,
    font=small,
    labelfont=bf
}

% ==================================================
% Literaturverzeichnis
\usepackage[
    backend=biber,
    style=authoryear,
    natbib=true
]{biblatex}
\addbibresource{literatur.bib}

% ==================================================
% Hyperlinks
\usepackage{hyperref}
\usepackage{xcolor}
\definecolor{links}{rgb}{0.36,0.54,0.66}
\hypersetup{
    colorlinks=true,
    linkcolor=black,
    urlcolor=links,
    citecolor=links,
    filecolor=links,
    pdfauthor={Dein Name},
    pdftitle={Titel der Arbeit},
    pdfsubject={Hausarbeit},
    pdfkeywords={Schlüsselwort1, Schlüsselwort2},
    pdfproducer={LaTeX},
    pdfcreator={pdfLaTeX}
}

% ==================================================
% Eigene Befehle (optional)
\newcommand{\R}{\mathbb{R}} % Beispiel für eigene Makros

\input{format}
\input{commands}

\begin{document}

\begin{Large}
    \textsf{\textbf{CPU Algorithm Design}} \\
    \\
    Exercise 3
\end{Large}
\vspace{1ex}
\textsf{\textbf{Students:}} \text{Vishal Mangukiya, Konstantin Benz} \\
\vspace{2ex}

\section*{3.1 Adapting reduce and transform}

The input containers in \texttt{reduce\_LoopUnrolling\_view.hpp} and \texttt{transform\_LoopUnrolling\_view.hpp} have been adapted as requested.
For the reduction routines, \texttt{std::views::repeat(1.0f, N)} is used to replace the original memory-backed containers. This ensures that the workload is compute-bound rather than memory-bound.
For the transform routines, \texttt{std::ranges::views::iota(0, N)} is used for the input range, and the output container \texttt{W} is a fixed-size \texttt{std::vector<Real>(256)} with modulo indexing.
\\
All adapted benchmark functions were successfully compiled and tested using the executables \texttt{reduceVbenchmarkUnroll} and \texttt{transformVbenchmarkUnroll} on the target system.

\pagebreak

\section*{3.2 Adapting benchTransformUnrollLoopPeelingDirective}

\pagebreak

\section*{3.3 Adapting benchReduceUnrollTreeDirective}

\pagebreak

\section*{3.4 Adapting benchReduceUnrollSimdXHorizontal and benchReduceUnrollSimdXVertical}

\pagebreak

\section*{3.5 Benchmarking}

\end{document}
% !TEX root = ./main.tex
% !TEX encoding = UTF-8 Unicode
\documentclass[a4paper,12pt]{article}

% ==================================================
% Sprache und Zeichencodierung
\usepackage[german]{babel}
\usepackage[utf8]{inputenc}
\usepackage[T1]{fontenc}

% ==================================================
% Seitenlayout
\usepackage[
  left=2.5cm,
  right=2.5cm,
  top=2.5cm,
  bottom=2.5cm
]{geometry}

% ==================================================
% Mathematik
\usepackage{amsmath}
\usepackage{amsfonts}
\usepackage{amssymb}
\usepackage{amsthm}
\usepackage{physics}
\usepackage{mathrsfs}
\usepackage{dsfont}
\usepackage{esint}

% ==================================================
% Programmierung
\usepackage{listings}

% ==================================================
% Textgestaltung
\usepackage{enumerate}
\usepackage[shortlabels]{enumitem}
\usepackage{framed}
\usepackage{csquotes} % wichtig für deutsche Anführungszeichen
\usepackage{microtype} % bessere Silbentrennung und Schriftbild

% ==================================================
% Tabellen, Grafiken
\usepackage{float}
\usepackage{tabularx}
\usepackage{multicol}
\usepackage{caption}
\usepackage{subcaption}
\captionsetup{
    format=hang,
    margin=10pt,
    font=small,
    labelfont=bf
}

% ==================================================
% Literaturverzeichnis
\usepackage[
    backend=biber,
    style=authoryear,
    natbib=true
]{biblatex}
\addbibresource{literatur.bib}

% ==================================================
% Hyperlinks
\usepackage{hyperref}
\usepackage{xcolor}
\definecolor{links}{rgb}{0.36,0.54,0.66}
\hypersetup{
    colorlinks=true,
    linkcolor=black,
    urlcolor=links,
    citecolor=links,
    filecolor=links,
    pdfauthor={Dein Name},
    pdftitle={Titel der Arbeit},
    pdfsubject={Hausarbeit},
    pdfkeywords={Schlüsselwort1, Schlüsselwort2},
    pdfproducer={LaTeX},
    pdfcreator={pdfLaTeX}
}

% ==================================================
% Eigene Befehle (optional)
\newcommand{\R}{\mathbb{R}} % Beispiel für eigene Makros

\usepackage{titlesec}
\usepackage[many]{tcolorbox}

% Adjust spacing after the chapter title
\titlespacing*{\chapter}{0cm}{-2.0cm}{0.50cm}
\titlespacing*{\section}{0cm}{0.50cm}{0.25cm}

% Indent 
\setlength{\parindent}{0pt}
\setlength{\parskip}{1ex}

% --- Theorems, lemma, corollary, postulate, definition ---
% \numberwithin{equation}{section}

\newtcolorbox{problem}[1][]{
    enhanced,
    colback = black!5,
    colbacktitle = black!5,
    coltitle = black,
    boxrule = 0pt,
    frame hidden,
    borderline west = {0.5mm}{0.0mm}{black},
    fonttitle = \bfseries\sffamily,
    breakable,
    before skip = 3ex,
    after skip = 3ex,
    sharp corners,
    fontupper = \raggedright,
    #1
}


\tcbuselibrary{skins, breakable}

% --- You can define your own color box. Just copy the previous \newtcbtheorm definition and use the colors of yout liking and the title you want to use.
% --- Basic commands ---
% Farben definieren
\definecolor{codegreen}{rgb}{0,0.6,0}
\definecolor{codegray}{rgb}{0.5,0.5,0.5}
\definecolor{codepurple}{rgb}{0.58,0,0.82}
\definecolor{backcolour}{rgb}{0.95,0.95,0.92}

%   Euler's constant
\newcommand{\eu}{\mathrm{e}}

%   Imaginary unit
\newcommand{\im}{\mathrm{i}}

%   Sexagesimal degree symbol
\newcommand{\grado}{\,^{\circ}}

% --- Inline C++ Code mit Syntax-Highlighting ---

% Listings-Style für C++
\lstdefinestyle{cppstyle}{
    language=C++,
    basicstyle=\ttfamily\small,
    keywordstyle=\color{blue},
    stringstyle=\color{codepurple},
    commentstyle=\color{codegreen},
    morecomment=[l][\color{magenta}]{\#},
    backgroundcolor=\color{backcolour},
    breaklines=true,
    breakatwhitespace=true,
    columns=fullflexible
}

% Inline C++ Command
\newcommand{\cppinline}[1]{\lstinline[style=cppstyle]{#1}}


% --- Comandos para álgebra lineal ---
% Matrix transpose
\newcommand{\transpose}[1]{{#1}^{\mathsf{T}}}

%%% Comandos para cálculo
%   Definite integral from -\infty to +\infty
\newcommand{\Int}{\int\limits_{-\infty}^{\infty}}

%   Indefinite integral
\newcommand{\rint}[2]{\int{#1}\dd{#2}}

%  Definite integral
\newcommand{\Rint}[4]{\int\limits_{#1}^{#2}{#3}\dd{#4}}

%   Dot product symbol (use the command \bigcdot)
\makeatletter
\newcommand*\bigcdot{\mathpalette\bigcdot@{.5}}
\newcommand*\bigcdot@[2]{\mathbin{\vcenter{\hbox{\scalebox{#2}{$\m@th#1\bullet$}}}}}
\makeatother

%   Hamiltonian
\newcommand{\Ham}{\hat{\mathcal{H}}}

%   Trace
\renewcommand{\Tr}{\mathrm{Tr}}

% Christoffel symbol of the second kind
\newcommand{\christoffelsecond}[4]{\dfrac{1}{2}g^{#3 #4}(\partial_{#1} g_{#2 #4} + \partial_{#2} g_{#1 #4} - \partial_{#4} g_{#1 #2})}

% Riemann curvature tensor
\newcommand{\riemanncurvature}[5]{\partial_{#3} \Gamma_{#4 #2}^{#1} - \partial_{#4} \Gamma_{#3 #2}^{#1} + \Gamma_{#3 #5}^{#1} \Gamma_{#4 #2}^{#5} - \Gamma_{#4 #5}^{#1} \Gamma_{#3 #2}^{#5}}

% Covariant Riemann curvature tensor
\newcommand{\covariantriemanncurvature}[5]{g_{#1 #5} R^{#5}{}_{#2 #3 #4}}

% Ricci tensor
\newcommand{\riccitensor}[5]{g_{#1 #5} R^{#5}{}_{#2 #3 #4}}

\begin{document}

\begin{Large}
    \textsf{\textbf{CPU Algorithm Design}} \\
    \\
    Exercise 1
\end{Large}
\vspace{1ex}
\textsf{\textbf{Students:}} \text{Vishal Mangukiya, Konstantin Benz} \\
\vspace{2ex}

\begin{problem}{}{Task 1.1.x}
    \begin{enumerate}[(1)]
        \item 
            You use \cppinline{std::optional<T>} to represent an object of type \cppinline{T} that may or may not be present. 
            It is a wrapper around the type \cppinline{T} that can be either empty or contain a value. 
            This is useful when you want to indicate that a value might not be available without using pointers or special 
            values (like \cppinline{nullptr} or \cppinline{-1}). \\
            For instance, a \cppinline{find_user(id)} function can return \cppinline{std::optional<User>-std::nullopt}
            if the user does not exist, otherwise the found \cppinline{User} object.
        \item 
            \cppinline{std::variant} is a type-safe union that can hold exactly one of several types. 
            A \cppinline{std::tuple} is a fixed-size collection of the same types.
            Therefore a \cppinline{std::variant} is more compact memory-wise than a \cppinline{std::tuple}, 
            because it only needs to store the size of the largest type, while a \cppinline{std::tuple} 
            needs to store the size of each type.
        \item  
            \cppinline{std::pair} and \cppinline{std::complex} give their two
            components a fixed semantic meaning (first/second,
            real/imaginary) and always contain exactly two elements.  
            In contrast, \cppinline{std::tuple} and \cppinline{std::array} are
            purely structural containers whose members are accessed by
            position and have no predefined interpretation.

        \item  
            \cppinline{std::pair} and \cppinline{std::tuple} can store
            heterogeneous types (each element may have a different
            type).  
            \cppinline{std::array<T, N>} and \cppinline{std::complex<T>}
            are homogeneous: every stored value has the same type
            \cppinline{T} (and, for \cppinline{std::complex}, there are always
            exactly two such values).

        \item  
            \cppinline{std::complex<T>} is a domain-specific numeric
            abstraction: beyond holding two values, it models the algebra
            of complex numbers and overloads arithmetic operators
            (`+`, `-`, `*`, `/`, `abs`, `arg`, …).
            The other templates are generic containers and provide no
            mathematical behaviour on their own.
    \end{enumerate}
\end{problem}

\begin{problem}{}{Task 1.2.x}
    \begin{enumerate}[(1)]
        \item Documented code including test inside file \cppinline{main-121.cpp}.
        \item Documented code including test inside file \cppinline{main-122.cpp}.
        \item Documented code including test inside file \cppinline{main-123.cpp}.
    \end{enumerate}
\end{problem}

\end{document}